%%%%%%%%%%%%%%%%%%%%%%%%%%%%%%%%%%%%
%                                  %
% Titre  : s_d.tex                 %
% Sujet  : Manuel de l'utilisateur %
%          du projet 'Scotch'      %
%          Distribution programmes %
% Auteur : Francois Pellegrini     %
%                                  %
%%%%%%%%%%%%%%%%%%%%%%%%%%%%%%%%%%%%

\section{Installation}
\label{sec-install}

Version {\sc \scotchver} of the \scotch\ software package is
distributed as free/libre software under the CeCILL-C free/libre
software license~\cite{cecill}, which is very similar to the GNU LGPL
license. Therefore, it is no longer distributed as a set of binaries,
but instead in the form of a source distribution, which can be
downloaded from the \scotch\ web page at
\url{http://www.labri.fr/~pelegrin/scotch/}~.
\\

All \scotch\ users are welcome to send an e-mail to the author so that
they can be added to the \scotch\ mailing list, and be automatically
informed of new releases and publications.
\\

The extraction process will create a {\tt scotch\_\scotchversub}
directory, containing several subdirectories and files. Please refer
to the files called {\tt LICENSE\_\lbt EN.txt} or
{\tt LICENCE\_\lbt FR.txt}, as well as file
{\tt INSTALL\_\lbt EN.txt}, to see under which conditions your
distribution of \scotch\ is licensed and how to install it.

\subsection{Thread issues}

To enable the use of POSIX threads in some routines, the {\tt
SCOTCH\_\lbt PTHREAD} flag must be set. If your MPI implementation is
not thread-safe, make sure this flag is not defined at compile time.

\subsection{File compression issues}

To enable on-the-fly compression and decompression of various formats,
the relevant flags must be defined. These flags are {\tt COMMON\_\lbt
FILE\_\lbt COMPRESS\_\lbt BZ2} for {\tt bzip2} (de)compression, {\tt
COMMON\_\lbt FILE\_\lbt COMPRESS\_\lbt GZ} for {\tt gzip}
(de)compression, and {\tt COMMON\_\lbt FILE\_\lbt COMPRESS\_\lbt LZMA}
for {\tt lzma} decompression. Note that the corresponding
development libraries must be installed on your system before compile
time, and that compressed file handling can take place only on systems
which support multi-threading or multi-processing. In the first case,
you must set the {\tt SCOTCH\_\lbt PTHREAD} flag in order to take
advantage of these features.

On Linux systems, the development libraries to install are {\tt
libbzip2\_1-\lbt devel} for the {\tt bzip2} format, {\tt zlib1-\lbt
devel} for the {\tt gzip} format, and {\tt liblzma0-\lbt devel} for
the {\tt lzma} format. The names of the libraries may vary according
to operating systems and library versions. Ask your system engineer in
case of trouble.

\subsection{Machine word size issues}
\label{sec-install-inttypesize}

The integer values handled by \scotch\ are based on the
{\tt SCOTCH\_\lbt Num} type, which equates by default to the {\tt int}
C type, corresponding to the {\tt INTEGER} Fortran type, both of which
being of machine word size. To coerce the length of the
{\tt SCOTCH\_\lbt Num} integer type to 32 or 64 bits, one can use the
``{\tt -DINTSIZE32}'' or ``{\tt -DINTSIZE64}'' flags, respectively, or
else use the ``{\tt -DINT=}'' definition, at compile time. For
instance, adding ``{\tt -DINT=long}'' to the {\tt CFLAGS} variable in
the {\tt Makefile.inc} file to be placed at the root of the source
tree will make all {\tt SCOTCH\_\lbt Num} integers become {\tt long} C
integers.

Whenever doing so, make sure to use integer types of equivalent length
to declare variables passed to \scotch\ routines from caller C and
Fortran procedures. Also, because of API conflicts, the
\metis\ compatibility library will not be usable. It is usually safer
and cleaner to tune your C and Fortran compilers to make them
interpret {\tt int} and {\tt INTEGER} types as 32 or 64 bit values,
than to use the aforementioned flags and coerce type lengths in your
own code.

Fortran users also have to take care of another size issue: since
there are no pointers in Fortran~77, the Fortran interface of some
routines converts pointers to be returned into integer indices with
respect to a given array (e.g. see
sections~\ref{sec-lib-func-graphdata},
\ref{sec-lib-func-meshdata}
and~\ref{sec-lib-func-geomdata}).
For 32\_64 architectures, such indices can be larger than the size of
a regular {\tt INTEGER}. This is why the indices to be returned are
defined by means of a specific integer type, {\tt SCOTCH\_Idx}. To
coerce the length of this index type to 32 or 64 bits, one can use the
``{\tt -DIDXSIZE32}'' or ``{\tt -DIDXSIZE64}'' flags, respectively, or
else use the ``{\tt -DIDX=}'' definition, at compile time. For
instance, adding ``{\tt -DIDX="long~long"}'' to the {\tt CFLAGS}
variable in the {\tt Makefile.inc} file to be placed at the root of
the source tree will equate all {\tt SCOTCH\_\lbt Idx} integers to
C {\tt long long} integers. By default, when the size of
{\tt SCOTCH\_\lbt Idx} is not explicitly defined, it is assumed to be
the same as the size of {\tt SCOTCH\_\lbt Num}.
